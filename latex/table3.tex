\begin{landscape}
\begin{longtable}[]{@{}
  >{\centering\arraybackslash}p{(\columnwidth - 8\tabcolsep) * \real{0.1}}
  >{\raggedleft\arraybackslash}p{(\columnwidth - 8\tabcolsep) * \real{0.25}}
  >{\raggedright\arraybackslash}p{(\columnwidth - 8\tabcolsep) * \real{0.2}}
  >{\raggedleft\arraybackslash}p{(\columnwidth - 8\tabcolsep) * \real{0.25}}
  >{\raggedright\arraybackslash}p{(\columnwidth - 8\tabcolsep) * \real{0.2}}@{}}
\caption{Checking FDO guidelines \autocite{boninoFAIRDigitalObject,fdo-RequirementSpec} against its current implementations as DOIP \autocite{foundationDigitalObjectInterface} and Linked Data Platform (LDP) \autocite{FDOFramework}, with suggestions for required additions.
\label{tbl:fdo-checks}}\tabularnewline
\toprule
\begin{minipage}[b]{\linewidth}\centering
\textbf{FDO Guideline}
\end{minipage} & \begin{minipage}[b]{\linewidth}\centering
DOIP 2.0
\end{minipage} & \begin{minipage}[b]{\linewidth}\centering
FDO suggestions
\end{minipage} & \begin{minipage}[b]{\linewidth}\centering
Linked Data Platform
\end{minipage} & \begin{minipage}[b]{\linewidth}\centering
LDP suggestion
\end{minipage} \\
\midrule
\endfirsthead
\toprule
\begin{minipage}[b]{\linewidth}\centering
\textbf{FDO Guideline}
\end{minipage} & \begin{minipage}[b]{\linewidth}\centering
DOIP 2.0
\end{minipage} & \begin{minipage}[b]{\linewidth}\centering
FDO suggestions
\end{minipage} & \begin{minipage}[b]{\linewidth}\centering
Linked Data Platform
\end{minipage} & \begin{minipage}[b]{\linewidth}\centering
LDP suggestion
\end{minipage} \\
\midrule
\endhead
G1: \emph{invest for many decades} & Handle system stable for 20 years, DOIP 2.0 since 2017. & Ensure FDO types will not be protocol-bound as DOIP might be updated/replaced & HTTP stable for 30 years, Semantic Web for 20 years. \texttt{http://} URIs replaced by \texttt{https://}. & Keep flexibility of RDF serialisation formats which may change more frequently \\
G2: \emph{trustworthiness} & DOI/Handle trusted by all major academic publishers and data repositories. DOIP relatively unknown, in effect only one implementation. & Further promote DOIP and justify its benefits. Build tutorials and OSI open source implementations. Standardise DOIP-over-HTTP alternative. & JSON-LD used by half of all websites \autocite{UsageStatisticsJSONLD}, however previous bad experiences with Semantic Web may deter adopters & Ensure simplicity for end developers, rather than semantic overengineering. Example-driven documentation. \\
G3: \emph{follows FAIR principles} & See Table \vref{tbl:fair-data-maturity-model} & Ensure all FAIR principles are covered, build complete examples. & Touched briefly, see Table \vref{tbl:fair-data-maturity-model} & Add explicit expression to show each FAIR pcinciple fulfilled. \\
G4: \emph{machine actionability} & CRUD and extension operations dynamically listed (see Table \vref{tbl:fdo-web-middleware}) & Specify which operations should work for a given type, to reduce need for dynamic lookup. Specify input/output expectations formally (e.g.~JSON Schema). & HTTP CRUD operations, Open API (see Table \vref{tbl:fdo-web-middleware}) & Document operations so client can make subsequent HTTP calls. \\
G5: \emph{abstraction principle} & Handle PIDs as abstraction base. DOIP operations can use any transport protocol. & Document transport protocols as FDOs, recommend which transport to use. & URI as abstraction base. Does not specify PID requirements. & Give stronger deployment recommendations. \\
G6: \emph{stable binding between entities} & Machine-navigation through PIDs and operations specified per type. Unclear when metadata field is a PID or plain text. & Make datatype of fields explicit to support navigation. & Machine-navigation through URIs via properties and types. Unclear when URI should be followed or is just identifier, but always distinct from plain text. & \\
G7: \emph{encapsulation} & Operations discovered at runtime (\texttt{0.DOIP/Op.ListOperations}). & Allow method discovery by type FDOs in advance (see PR-TypingFDOs-2.0-20220608). & HTTP methods discovered at runtime (\texttt{OPTIONS}), indempotent methods attempted directly. Unsupported methods reported using LDP constraints to human-readable text. & Declare supported methods in advance, e.g.~OpenAPI \autocite{OpenAPISpecificationV3} \\
G8: \emph{technology independence} & In theory independent, in reality depends on single implementations of Handle system and DOIP & Encourage open source DOIP testbeds and lighter reference implementations & Multiple HTTP implementations, multiple LDP implementations. No FDOF implementations. & Develop demonstrator of FDOF usage based on existing LDP server. \\
G9: \emph{standard compliance} & Handle \autocite{rfc3650}, DOIP \autocite{foundationDigitalObjectInterface}. FDO requirements not standardised yet. & Formalise standard process of FDO requirements \autocite{fdo-DocProcessStd} & HTTP, LDP. FDOF not yet standardised & Formalise FDOF from FDOF-SEM working group \\
FDOF1: \emph{PID as basis} & Extensive use of Handle system. & Clarify how local testing handles can be used during development, how ``temporary'' FDOs should evolve \autocite{fdo-PIDProfileAttributes}. Register \texttt{0.DOIP/*} and \texttt{0.FDO/*} as PIDs. & HTTP URLs as basis for identifiers, but they are frequently not persistent. & Add strong guidance for PID services like w3id and persistence policies. \\
FDOF2: \emph{PID record w/ type} & Unspecified how to resolve from Handle to DOIP Service (!), in practice \texttt{10320/loc}, \texttt{0.TYPE/DOIPService}, \texttt{URL}, \texttt{URL\_REPLICA} & Document requirements for PID Record & w3id/purl PIDs redirect without giving any metadata. Datacite DOIs content-negotiate to give registered metadata. & Add FAIR Signposting at PID provider for minimal PID record \\
FDOF3: \emph{PID resolvable to bytestream \& metadata} & Byte stream resolvable (\texttt{0.DOIP/Retrieve}), \texttt{includeElementData} option can retrieve bytestream or full object structure. No method/attribute defined for separate metadata, only directly in PID Record. Unclear meaning of multiple items and bytestream chunks. & Clarify expectations for multiple items. Recommend chunks to not be used. & URIs resolvable by default. Multiple ways to resolve metadata, unclear preference. & Add FAIR Signposting and preference order. \\
FDOF4: \emph{Additional attributes} & Freetext attribute keys. Attributes should be defined for FDO type (?). & Require that attribute keys should be PIDs (or have predefined mapping to PIDs). Explicitly allow attributes not already defined in type. & All attributes individually identified. Any Linked Data attributes can be used by URI or with mapped prefix. & Clarify type expectations of required/recommended/optional attributes. \\
FDOF5: \emph{Interface: operation by PID} & Extended operations use PID, but ``pid-like'' DOIP operations/types are not registered as handles. & Register \texttt{0.DOIP/*} and \texttt{0.FDO/*} as PIDs. Clarify that operations can be mapped to protocol directly. & CRUD operations used directly in HTTP (e.g.~\texttt{PUT}). Unclear how to provide PID for additional operations. & Specify how additional operations should be called over HTTP. \\
FDOF6: \emph{CRUD operations + extensions} & \texttt{0.DOIP/Op.Create}, \texttt{Op.Retrieve}, \texttt{Op.Update}, \texttt{Op.Delete} but also \texttt{0.DOIP/Op.Search}. & Document & \texttt{PUT}, \texttt{GET}, \texttt{POST}, \texttt{DELETE}, \texttt{PATCH}, \texttt{HEAD} -- extension operations (e.g.~WebDAV \texttt{COPY}) not used, resource patterns \autocite{martinekuanWebAPIDesign} are used instead. & Document how operation resources can be discovered from an LPD container. Document search API. \\
FDOF7: \emph{FDOF Types related to operations} & Not yet formalised, by DOIP discoverable on a given FDO rather than type. PR-TypingFDOs leaves this open. & Add explicit relation between type and operations & \texttt{OPTIONS} per LDP Resource, but not by type. Common types (\texttt{ldp:Resource}, \texttt{ldp:Container}) indicate LDP support, but are not required. & Always make LDP types explicit in FDO profile. \\
FDOF8: \emph{Metadata as FDO, semantic assertions} & DOIP includes all metadata in PID Record. Separate Metadata FDO need custom property. & Specify a \texttt{0.FDO/metadata} or similar to point to Metadata FDOs. & Assertions are always with semantics, using RDF vocabularies. Unspecified how to find additional metadata resources, \texttt{rdfs:seeAlso} is common. & Use FAIR Signposting \texttt{describedby} link relation to additional metadata PIDs \\
FDOF9: \emph{Different metadata levels} & Defines open-ended ``Response Attributes'' without namespaces, but mandated as ``None'' for all CRUD operations. Metadata would need to be bundled within custom FDO types or attributes. Unclear how levels are separated within single FDO representation (need FDOF8?). & Declare which metadata are expected within response attribute or within FDO object. Require PIDs for custom attributes. Define how alternate metadata levels can be represented separately. & Undefined how to handle multiple metadata granularities or domains, alternative LDP containers can present different views on same stored objects. & Define how to navigate to alternate views and their semantic implications, e.g.~\texttt{owl:sameAs} \\
FDOF10: \emph{Metadata schemas by community} & Metadata schemas are in practice managed on single CORDA server as local types, using JSON Schema. & Require types to be FDOs with registered PIDs, implement shared types. & Plethora of existing RDF vocabularies/ontologies managed by larger communities, e.g.~\href{https://obofoundry.org/}{OBO Foundry} \autocite{smithOBOFoundryCoordinated2007a} & Rather document better how individual ad-hoc schemas can be started for prototypes. \\
FDOF11: \emph{FDO collections w/ semantic relations} & Collection type undefined by DOIP. Informal use of \texttt{HAS\_PARTS} Handle attribute (e.g. \autocite{DataInformationView}). & & LDP Containers required by specification, also user-created (eg. \texttt{BasicContainer}). & Clarify relation to other collections like DCAT 3 \autocite{w3-vocab-dcat-3}, \href{https://schema.org/Dataset}{Schema.org Dataset}, OAI-ORE \autocite{ORESpecificationAbstract} \\
FDOF12: \emph{Deleted FDO preserve PID w/ tombstone} & Tombstone for deleted resource undefined by DOIP. \texttt{0.DOIP/Status.104} status code does not distinguish ``Not Found'' or ``Gone'' & Formalise tombstone requirements with new FDO type & \texttt{410\ Gone} recommended, but \texttt{404\ Not\ Found} common. No requirement for tombstone serialisation & Formalise tombstone requirements and serialisation \\
\bottomrule
\end{longtable}
\end{landscape}